\documentclass{paper}
\usepackage{amsmath}
\usepackage{amssymb}
\DeclareMathOperator*{\argmin}{arg\,min}

\begin{document}

\title{Merging two Gaussian distributions}
\maketitle

We want to merge two Gaussians over $x$ and $v$ into one over $v$

\begin{equation}
p(x \mid v,z) p(v \mid g) = \mathcal{N}(x;zAv,\sigma_x I) \mathcal{N}(v;0,C_v)
\end{equation}
%
The Gaussian over $x$ spelled out is

\begin{equation} 
\mathcal{N}(x;zAv,\sigma_x I) = \sqrt{\frac{1}{(2\pi)^{Dx} \sigma_x^{Dx}}}e^{-\frac{1}{2 \sigma_x} (x-zAv)^T(x-zAv)}
\end{equation}

\begin{equation} 
-\frac{1}{2} (x-zAv)^T(x-zAv) = -\frac{1}{2} (x^Tx - zv^TA^Tx - zxAv + z^2 v^TA^TAv)
\end{equation}
%
as $v^TA^Tx = (xAv)^T$, and both are scalars, thus equal to their transposes, it's also true that $v^TA^Tx = xAv$

\begin{equation} 
\begin{split}
-\frac{1}{2} (x-zAv)^T(x-zAv) = -\frac{1}{2} (x^Tx - 2zxAv + z^2 v^TA^TAv) = \\
= -\frac{x^Tx}{2} + zxAv -\frac{z^2}{2} v^TA^TAv
\end{split}
\end{equation}
%
we have the identity for any symmetric matrix $M$ and vector $b$ that

\begin{equation} 
-\frac{1}{2} v^T M v + b^Tv = -\frac{1}{2} (v - M^{-1}b)^T M (v - M^{-1}b) + \frac{1}{2}b^T M^{-1} b
\end{equation}
%
making the substitution $M = z^2A^TA$ and $b = (zxA)^T=zA^Tx^T$, yielding $M^{-1} = \frac{1}{z^2}(A^TA)^{-1}$ and $M^{-1} b = \frac{1}{z}(A^TA)^{-1}A^Tx^T = \frac{1}{z}A^{+}x^T$, where $A^{+}$ is the Moore-Penrose pseudoinverse of $A$. Thus we get

\begin{equation}
\begin{split}
-\frac{1}{2} (x-zAv)^T(x-zAv) = \\
= -\frac{x^Tx}{2} -\frac{1}{2} (v - \frac{1}{z}A^{+}x^T)^T z^2A^TA (v - \frac{1}{z}A^{+}x^T)  + \frac{1}{2} (A^Tx^T)^T (A^TA)^{-1}A^Tx^T = \\
=-\frac{x^Tx}{2} -\frac{1}{2} (v - \frac{1}{z}A^{+}x^T)^T z^2A^TA (v - \frac{1}{z}A^{+}x^T)  + \frac{1}{2} x A A^{-1} A^{-T} A^T x^T = \\
-\frac{x^Tx}{2} -\frac{1}{2} (v - \frac{1}{z}A^{+}x^T)^T z^2A^TA (v - \frac{1}{z}A^{+}x^T)  + \frac{1}{2} (x A A^{-1} A^{-T} A^T x^T)^T = \\
-\frac{x^Tx}{2} -\frac{1}{2} (v - \frac{1}{z}A^{+}x^T)^T z^2A^TA (v - \frac{1}{z}A^{+}x^T)  + \frac{x^Tx}{2} = \\
 -\frac{1}{2} (v - \frac{1}{z}A^{+}x^T)^T z^2A^TA (v - \frac{1}{z}A^{+}x^T)
\end{split}
\end{equation}
%
which implies

\begin{equation}
e^{-\frac{1}{2 \sigma_x} (x-zAv)^T(x-zAv)} = e^{-\frac{1}{2} (v - \frac{1}{z}A^{+}x^T)^T \frac{z^2}{\sigma_x} A^TA(v - \frac{1}{z}A^{+}x^T)}
\end{equation}
% 
meaning that

\begin{equation} 
\mathcal{N}(x;zAv,\sigma_x I) = \alpha \mathcal{N}(v;\frac{1}{z}A^{+}x^T,\frac{\sigma_x}{z^2} (A^TA)^{-1})
\end{equation}
%
and as the formulas in the exponents are equal, the constant  $\alpha$ is given by the ratio of the normalisation terms

\begin{eqnarray}
\sqrt{\frac{1}{(2\pi)^{Dx} \sigma_x^{Dx}}} = \alpha \sqrt{\frac{1}{(2\pi)^{Dv} \det(\frac{\sigma_x}{z^2} (A^TA)^{-1})}} \\
\alpha = \sqrt{\frac{ (2\pi)^{Dv} \frac{\sigma_x^{D_v}}{z^{2D_v}} \det( (A^TA)^{-1}) }{ (2\pi)^{Dx} \sigma_x^{Dx} }} \\
\alpha = \sqrt{\frac{ (2\pi)^{Dv} \sigma_x^{D_v} }{ (2\pi)^{Dx} \sigma_x^{Dx} z^{2D_v} \det(A^TA)}}
\end{eqnarray}
%
making the simplifying assumption $D_x = D_v$ we arrive to

\begin{equation} 
\mathcal{N}(x;zAv,\sigma_x I) = \frac{1}{\sqrt{\det(A^TA)}} \frac{1}{z^{D_v}} \mathcal{N}(v;\frac{1}{z}A^{+}x^T,\frac{\sigma_x}{z^2} (A^TA)^{-1})
\end{equation}
%
we can merge two Gaussian distributions over $v$ into one by using the following formula

\begin{equation} 
\mathcal{N}(v;\mu_1,C_1) \mathcal{N}(v;\mu_2,C_2) = \mathcal{N}(\mu_1;\mu_2,C_1 + C_2) \mathcal{N}(v; \mu_c,C_c)
\end{equation}
%
where $C_c = (C_1^{-1} + C_2^{-1})^{-1}$ and $\mu_c = C_c (C_1^{-1}\mu_1 + C_2^{-1}\mu_2)$

\end{document}