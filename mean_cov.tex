\documentclass{paper}
\usepackage{amsmath}
\usepackage{amssymb}
\DeclareMathOperator*{\argmin}{arg\,min}

\begin{document}

\title{Latent variables affecting mean and covariace}
\maketitle

\begin{eqnarray}
p(g) = \textrm{Gam}(g; \alpha_g,\theta_g) \\
p(z) = \textrm{Gam}(z; \alpha_z,\theta_z) \\
p(x \mid v,z) = \mathcal{N}(x; zAv,\sigma_x I) \\
p_c(v \mid g) = \mathcal{N}(v; 0,\sum_{k=1}^K g_k C_k) \\
p_m(v \mid g) = \mathcal{N}(v; Bg,\sigma_v I)
\end{eqnarray}

\begin{equation}
p(v \mid x) = \iint_{-\infty}^\infty p(v \mid x,g,z) p(g) p(z) \mathrm{d}g \mathrm{d}z
\end{equation}
%
we can do this by either numerical integration or sampling

\begin{eqnarray}
p(v \mid x) \approx \sum_{z \in \left[z_{min} z_{max} \right]}  \sum_{g \in \left[g_{min} g_{max} \right]} p(v \mid x,g,z) p(g) p(z) \\
p(v \mid x) \approx \frac{1}{L} \sum_{l=1}^L p(v \mid x,g^l,z^l)
\end{eqnarray}

either way, we approximate the marginal posterior with a finite mixture. In case of sampling, of $L$ components, for wich the covariance is given in the following form

\begin{equation}
C_{v \mid x} ~ \approx \frac{1}{L} \sum_{l=1}^L C_{v \mid xgz}^l + ( \mu_{v \mid xgz}^l - \frac{1}{L} \sum_{m=1}^L \mu_{v \mid xgz}^m) ( \mu_{v \mid xgz}^l - \frac{1}{L} \sum_{m=1}^L \mu_{v \mid xgz}^m)^T
\end{equation}

\begin{eqnarray}
p_c(v \mid x,g,z) = \mathcal{N}(v;\mu_c,C_c) \\
C_c =  \left(\frac{z^2}{\sigma_x} A^T A + \left[\sum_{k=1}^K g_k C_k \right]^{-1}\right)^{-1} \\
\mu_c = \frac{z}{\sigma_x} C_c A^T x \\
p_c(m \mid x,g,z) = \mathcal{N}(v;\mu_m,C_m) \\
C_m =  \left(\frac{z^2}{\sigma_x} A^T A + \frac{1}{\sigma_v} I \right)^{-1} \\
\mu_m = C_m \left(\frac{z}{\sigma_x} A^T x +  \frac{1}{\sigma_v} B g \right)
\end{eqnarray}


\end{document}
