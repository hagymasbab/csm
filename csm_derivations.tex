\documentclass{paper}
\usepackage{amsmath}
\usepackage{amssymb}
\usepackage{hyperref}
\hypersetup{pdftex,colorlinks=true,allcolors=blue}
\DeclareMathOperator*{\argmin}{arg\,min}
\DeclareMathOperator*{\argmax}{arg\,max}

\begin{document}

\title{Derivations for the CSM model}
\maketitle

\section{Definition of the model}

A gestalt, a perceptual object, is characterised by a covariance component for the joint distribution of visual neural activity. 

\begin{eqnarray}
p(v \mid g) = \mathcal{N}(v; 0,C_v) \\
C_v = \sum_{k=1}^K g_k C_k \label{eq:cv}
\end{eqnarray}
%
where K is the fixed number of possible gestalts in the visual scene and $g_k$ is the strength of the gestalt number $k$, coming from a $K$-dimensional Gamma prior distribution with shape and scale  parameters $\alpha_g$ and $\theta_g$ controlling the sparsity of the prior.

\begin{equation}
p(g) = \textrm{Gam}(g; \alpha_g,\theta_g)
\end{equation}
%
The global contrast of the image patch is encoded by a scalar variable $z$, also coming from a Gamma prior

\begin{equation}
p(z) = \textrm{Gam}(z; \alpha_z,\theta_z)
\end{equation}

The pixel intensities are generated from the neural activity through a set of linear projective field models, possibly Gabor filters, $A$, scaled by the contrast and adding some independent observational noise.

\begin{equation}
p(x \mid v,z) = \mathcal{N}(x; zAv,\sigma_x I)
\end{equation}

\section{Likelihoods}

\subsection{Likelihood of $g$ and $z$}

by intuition:

\begin{equation} \label{eq:like_gz_int}
p(x \mid z,g) = \mathcal{N}(x;0,\sigma_x I + z^2 A \left( \sum_{k=1}^K g_k U_k^T U_k \right)A^T)
\end{equation}

by algebraic derivation:

\begin{eqnarray}
p(x \mid z,g) = \int_{-\infty}^{\infty} p(x,v \mid z,g)  \mathrm{d}v = \frac{1}{z^{D_v} \sqrt{\det(A^TA)}} \mathcal{N}(\frac{1}{z}A^{+}x;0,\frac{\sigma_x}{z^2} (A^TA)^{-1} + \sum_{k=1}^K g_k U_k^T U_k) \\
f(x,z) \equiv \frac{1}{z}A^{+}x, ~ h(z) \equiv  \frac{1}{z^{D_v} } \\
C(z,g) \equiv \frac{\sigma_x}{z^2} (A^TA)^{-1} + \sum_{k=1}^K g_k U_k^T U_k \\
p(x \mid z,g) = \frac{1}{\sqrt{\det(A^TA)}} h(z) \mathcal{N}(f(x,z);0,C(z,g)) \label{eq:like_gz_alg}
\end{eqnarray}

\subsection{Log-likelihood of the parameters}

\begin{equation}
p(X \mid C_{1..K}) = \prod_{n=1}^N p(x_n \mid C_{1..K}) 
\end{equation}

\begin{eqnarray}
p(X \mid C_{1..K}) = \prod_{n=1}^N \iint_{-\infty}^{\infty} p(x_n \mid z,g) p(g) p(z) \mathrm{d}g\mathrm{d}z \\
\end{eqnarray}
%
approximation of the integrals by samples from the priors $p(g_n)$ and $p(z_n)$

\begin{equation}
p(X \mid C_{1..K}) \approx \prod_{n=1}^N \frac{1}{L} \sum_{l=1}^{L}p(x_n \mid z^l,g^l)
\end{equation}

using \ref{eq:like_gz_int}

\begin{eqnarray}
\log p(X \mid C_{1..K}) \approx -N \log L +\sum_{n=1}^N \log \sum_{l=1}^{L} \mathcal{N}(x_n;0,\sigma_x I + z^2 A \left( \sum_{k=1}^K g_k U_k^T U_k \right)A^T) \\
C_{int} = \sigma_x I + z^2 A \left( \sum_{k=1}^K g_k U_k^T U_k \right)A^T \\
\mathcal{L}^{int}_n = \sum_{l=1}^{L} \mathcal{N}(x_n;0,C_{int}) \\
\log p(X \mid C_{1..K}) \approx -N \log L +\sum_{n=1}^N \log \mathcal{L}^{int}_n
\end{eqnarray}

using \ref{eq:like_gz_alg}

\begin{eqnarray}
p(X \mid C_{1..K}) \approx \left( L \sqrt{\det(A^TA)} \right)^{-N} \prod_{n=1}^N \sum_{l=1}^{L} h(z^l) \mathcal{N}(f(x_n,z^l);0,C(z^l,g^l)) \\
\log p(X \mid C_{1..K}) \approx -N (\log L + \frac{1}{2}\log(\det(A^TA))) + \sum_{n=1}^N \log \left[ \sum_{l=1}^{L} h(z^l) \mathcal{N}(f(x_n,z^l);0,C(z^l,g^l)) \right] \\
h^l \equiv h(z^l), ~ f_n^l \equiv f(x_n,z^l), ~ C^l \equiv C(z^l,g^l) \\
\mathcal{L}_n^l \equiv h^l \mathcal{N}(f_n^l;0,C^l), ~ \mathcal{L}_n \equiv  \sum_{l=1}^{L} \mathcal{L}_n^l \\
\log p(X \mid C_{1..K}) \approx -N (\log L + \frac{1}{2}\log(\det(A^TA))) + \sum_{n=1}^N \log \mathcal{L}_n
\end{eqnarray}

\subsubsection{Derivative w.r.t. $U_{1\dots K}$ based on \ref{eq:like_gz_alg}}

\begin{equation}
\begin{split}
\frac{\partial \log p(X \mid C_{1..K})}{\partial \left[ U_k \right]_{i,j}} \approx \sum_{n=1}^N \frac{\partial \log \mathcal{L}_n}{\partial \left[ U_k \right]_{i,j}} = \sum_{n=1}^N \frac{1}{\mathcal{L}_n} \frac{\partial \mathcal{L}_n}{\partial \left[ U_k \right]_{i,j}} = \sum_{n=1}^N \frac{1}{\mathcal{L}_n}  \sum_{l=1}^{L} \frac{\partial \mathcal{L}_n^l}{\partial \left[ U_k \right]_{i,j}} = \\
=  \sum_{n=1}^N \frac{1}{\mathcal{L}_n}  \sum_{l=1}^{L} \textrm{Tr} \left[ \frac{\partial \mathcal{L}_n^l}{\partial C^l} \frac{\partial C^l}{\partial \left[ U_k \right]_{i,j}} \right]
\end{split}
\end{equation}
% 
the derivatives in this formula are the following

\begin{eqnarray}
\begin{split}
\frac{\partial \mathcal{L}_n^l}{\partial C^l} = h^l \frac{\partial}{\partial C^l} \mathcal{N}(f_n^l;0,C^l) = h^l \mathcal{N}(f_n^l;0,C^l) \frac{\partial}{\partial C^l} \log \mathcal{N}(f_n^l;0,C^l) = \\
= -\frac{h^l}{2} \mathcal{N}(f_n^l;0,C^l) \left[ (C^l)^{-1} - (C^l)^{-1} f_n^l (f_n^l)^T (C^l)^{-1} \right]
\end{split} \\
\frac{\partial C(z,g)}{\partial \left[ U_k \right]_{i,j}} = g_k \frac{\partial \left( U_k^T U_k \right)}{\partial \left[ U_k \right]_{i,j}} = g_k \left( U_k^T J^{ij} + J^{ji} U_k \right) \equiv g_k \hat U_k^{ij}
\end{eqnarray}
%
substituting back to the derivative

\begin{equation}
\begin{split}
\frac{\partial \log p(X \mid C_{1..K})}{\partial \left[ U_k \right]_{i,j}} \approx \\
\approx  -\frac{1}{2} \sum_{n=1}^N \frac{1}{\mathcal{L}_n}  \sum_{l=1}^{L} h^l g^l_k \mathcal{N}(f_n^l;0,C^l)  \textrm{Tr} \left[ \left[ (C^l)^{-1} - (C^l)^{-1} f_n^l (f_n^l)^T (C^l)^{-1} \right] \hat U_k^{ij} \right] = \\
= -\frac{1}{2} \textrm{Tr} \left[ \sum_{n=1}^N \left(\frac{1}{\mathcal{L}_n}  \sum_{l=1}^{L} h^l g^l_k \mathcal{N}(f_n^l;0,C^l)   \left[ (C^l)^{-1} - (C^l)^{-1} f_n^l (f_n^l)^T (C^l)^{-1} \right] \right) \hat U_k^{ij} \right]
\end{split}
\end{equation}
%
The regularities of the $\hat U_k$ matrices allow us to replace the trace with a much more efficient computation:

\begin{eqnarray}
M_k = -\frac{1}{2} \sum_{n=1}^N \left(\frac{1}{\mathcal{L}_n}  \sum_{l=1}^{L} h^l g^l_k \mathcal{N}(f_n^l;0,C^l)   \left[ (C^l)^{-1} - (C^l)^{-1} f_n^l (f_n^l)^T (C^l)^{-1} \right] \right) \\
\frac{\partial \log p(X \mid C_{1..K})}{\partial \left[ U_k \right]_{i,j}} \approx \textrm{Tr} \left[M_k \hat U_k^{ij} \right] = \sum_{a=1}^{Dv} \left[ M_k \right]_{j,a} \left[ U_k \right]_{i,a} + \left[ M_k \right]_{a,j} \left[ U_k \right]_{i,a}
\end{eqnarray}

\subsubsection{Derivative w.r.t. $U_{1\dots K}$ based on \ref{eq:like_gz_int}}

\begin{equation}
\begin{split}
\frac{\partial \log p(X \mid C_{1..K})}{\partial \left[ U_k \right]_{i,j}} \approx \sum_{n=1}^N \frac{\partial \log \mathcal{L}^{int}_n}{\partial \left[ U_k \right]_{i,j}} = \sum_{n=1}^N \frac{1}{\mathcal{L}^{int}_n} \frac{\partial \mathcal{L}^{int}_n}{\partial \left[ U_k \right]_{i,j}} = \sum_{n=1}^N \frac{1}{\mathcal{L}^{int}_n}  \sum_{l=1}^{L} \frac{\partial \mathcal{N}(x_n;0,C_{int})}{\partial \left[ U_k \right]_{i,j}} = \\
=  \sum_{n=1}^N \frac{1}{\mathcal{L}_n}  \sum_{l=1}^{L} \textrm{Tr} \left[ \frac{\partial \mathcal{N}(x_n;0,C_{int})}{\partial C_{int}} \frac{\partial C_{int}}{\partial \left[ U_k \right]_{i,j}} \right]
\end{split}
\end{equation}

\begin{eqnarray}
\begin{split}
\frac{\partial \mathcal{N}(x_n;0,C_{int})}{\partial  C_{int}} = \frac{\partial}{\partial  C_{int}} \mathcal{N}(f_n^l;0, C_{int}) = \mathcal{N}(f_n^l;0, C_{int}) \frac{\partial}{\partial  C_{int}} \log \mathcal{N}(f_n^l;0, C_{int}) = \\
= -\frac{1}{2} \mathcal{N}(f_n^l;0, C_{int}) \left[ ( C_{int})^{-1} - ( C_{int})^{-1} f_n^l (f_n^l)^T ( C_{int})^{-1} \right]
\end{split} \\
\frac{\partial C(z,g)}{\partial \left[ U_k \right]_{i,j}} = g_k \frac{\partial \left( U_k^T U_k \right)}{\partial \left[ U_k \right]_{i,j}} = g_k \left( U_k^T J^{ij} + J^{ji} U_k \right) \equiv g_k \hat U_k^{ij}
\end{eqnarray}

\subsection{Complete-data log-likelihood}

\subsubsection{Expectation w.r.t. the posterior}
\subsubsection{Derivative w.r.t. $C_{1\dots K}$}


\section{Posteriors}

\subsection{Full posterior}

\begin{equation}
p(v,g,z \mid x) = p(x \mid v,g,z) p(v \mid g) p(g) p(z) \frac{1}{p(x)} \sim  p(x \mid v,z) p(v \mid g) p(g) p(z)
\end{equation}
%
so the log-posterior will be the following, up to an additive constant, using Gamma priors over $g$ and $z$ defined by shape and scale parameters:

\begin{equation}
\begin{split}
\log p(v,g,z \mid x) \sim \log p(x \mid v,z) + \log p(v \mid g) + \log p(g) + \log p(z) = \\
= \log \mathcal{N}(x;zAv,\sigma_x I) + \log \mathcal{N}(v;0,C_v) + \log \textrm{Gam}(g;sh_g,sc_g) + \log \textrm{Gam}(z;sh_z,sc_z)
\end{split}
\end{equation}
%
logarithms of the used pdfs look as follows:

\begin{eqnarray}
\log  \mathcal{N}(y;\mu,C) = -\frac{1}{2} \left[ \log (2\pi) + \log \det (C) + (y - \mu)^T C^{-1} (y-\mu) \right] \\
\log  \textrm{Gam}(y;sh,sc) = \log(1) - \log(\Gamma(sh)) - sh \log(sc) + (sh-1) \log(y) - \frac{y}{sc}
\end{eqnarray}

\subsubsection{Derivative w.r.t. $v$}
\subsubsection{Derivative w.r.t. $g$}
\subsubsection{Derivative w.r.t. $z$}


\subsection{Marginal posterior of $g$}
\subsubsection{MAP estimate of $g$}

\begin{eqnarray}
g_{MAP} = \argmax_g p(g \mid x) = \argmax_g \frac{p(x \mid g)p(g)}{p(x)} = \argmax_g p(x \mid g)p(g)\\
p(x \mid g) = \int_{-\infty}^{\infty} p(x \mid z,g)p(z)\mathrm{d}z \approx \frac{1}{L} \sum_{l=1}^{L} p(x \mid g,z^l)
\end{eqnarray}


\section*{Appendix}

\subsection{Merging two Gaussian distributions}

We want to merge two Gaussians over $x$ and $v$ into one over $v$

\begin{equation}
p(x \mid v,z) p(v \mid g) = \mathcal{N}(x;zAv,\sigma_x I) \mathcal{N}(v;0,C_v)
\end{equation}
%
The Gaussian over $x$ spelled out is

\begin{equation} 
\mathcal{N}(x;zAv,\sigma_x I) = \sqrt{\frac{1}{(2\pi)^{Dx} \sigma_x^{Dx}}}e^{-\frac{1}{2 \sigma_x} (x-zAv)^T(x-zAv)}
\end{equation}

\begin{equation} 
-\frac{1}{2} (x-zAv)^T(x-zAv) = -\frac{1}{2} (x^Tx - zv^TA^Tx - zx^TAv + z^2 v^TA^TAv)
\end{equation}
%
as $v^TA^Tx = (x^TAv)^T$, and both are scalars, thus equal to their transposes, it's also true that $v^TA^Tx = x^TAv$

\begin{equation} 
\begin{split}
-\frac{1}{2} (x-zAv)^T(x-zAv) = -\frac{1}{2} (x^Tx - 2zx^TAv + z^2 v^TA^TAv) = \\
= -\frac{x^Tx}{2} + zx^TAv -\frac{z^2}{2} v^TA^TAv
\end{split}
\end{equation}
%
we have the identity for any symmetric matrix $M$ and vector $b$ that

\begin{equation} 
-\frac{1}{2} v^T M v + b^Tv = -\frac{1}{2} (v - M^{-1}b)^T M (v - M^{-1}b) + \frac{1}{2}b^T M^{-1} b
\end{equation}
%
making the substitution $M = z^2A^TA$ and $b = (zx^TA)^T=zA^Tx$, yielding $M^{-1} = \frac{1}{z^2}(A^TA)^{-1}$ and $M^{-1} b = \frac{1}{z}(A^TA)^{-1}A^Tx = \frac{1}{z}A^{+}x$, where $A^{+}$ is the Moore-Penrose pseudoinverse of $A$. Thus we get

\begin{equation}
\begin{split}
-\frac{1}{2} (x-zAv)^T(x-zAv) = \\
= -\frac{x^Tx}{2} -\frac{1}{2} (v - \frac{1}{z}A^{+}x)^T z^2A^TA (v - \frac{1}{z}A^{+}x)  + \frac{1}{2} (A^Tx)^T (A^TA)^{-1}A^Tx = \\
=-\frac{x^Tx}{2} -\frac{1}{2} (v - \frac{1}{z}A^{+}x)^T z^2A^TA (v - \frac{1}{z}A^{+}x)  + \frac{1}{2} x^T A A^{-1} A^{-T} A^T x = \\
= -\frac{x^Tx}{2} -\frac{1}{2} (v - \frac{1}{z}A^{+}x)^T z^2A^TA (v - \frac{1}{z}A^{+}x)  + \frac{x^Tx}{2} = \\
= -\frac{1}{2} (v - \frac{1}{z}A^{+}x)^T z^2A^TA (v - \frac{1}{z}A^{+}x)
\end{split}
\end{equation}
%
which implies

\begin{equation}
e^{-\frac{1}{2 \sigma_x} (x-zAv)^T(x-zAv)} = e^{-\frac{1}{2} (v - \frac{1}{z}A^{+}x^T)^T \frac{z^2}{\sigma_x} A^TA(v - \frac{1}{z}A^{+}x)}
\end{equation}
% 
meaning that

\begin{equation} 
\mathcal{N}(x;zAv,\sigma_x I) = \alpha \mathcal{N}(v;\frac{1}{z}A^{+}x,\frac{\sigma_x}{z^2} (A^TA)^{-1})
\end{equation}
%
and as the formulas in the exponents are equal, the constant  $\alpha$ is given by the ratio of the normalisation terms

\begin{eqnarray}
\sqrt{\frac{1}{(2\pi)^{Dx} \sigma_x^{Dx}}} = \alpha \sqrt{\frac{1}{(2\pi)^{Dv} \det(\frac{\sigma_x}{z^2} (A^TA)^{-1})}} \\
\alpha = \sqrt{\frac{ (2\pi)^{Dv} \frac{\sigma_x^{D_v}}{z^{2D_v}} \det( (A^TA)^{-1}) }{ (2\pi)^{Dx} \sigma_x^{Dx} }} \\
\alpha = \sqrt{\frac{ (2\pi)^{Dv} \sigma_x^{D_v} }{ (2\pi)^{Dx} \sigma_x^{Dx} z^{2D_v} \det(A^TA)}}
\end{eqnarray}
%
making the simplifying assumption $D_x = D_v$ we arrive to

\begin{equation} 
\mathcal{N}(x;zAv,\sigma_x I) = \frac{1}{\sqrt{\det(A^TA)}} \frac{1}{z^{D_v}} \mathcal{N}(v;\frac{1}{z}A^{+}x,\frac{\sigma_x}{z^2} (A^TA)^{-1})
\end{equation}
%
we can merge two Gaussian distributions over $v$ into one by using the following formula

\begin{equation} 
\mathcal{N}(v;\mu_1,C_1) \mathcal{N}(v;\mu_2,C_2) = \mathcal{N}(\mu_1;\mu_2,C_1 + C_2) \mathcal{N}(v; \mu_c,C_c)
\end{equation}
%
where $C_c = (C_1^{-1} + C_2^{-1})^{-1}$ and $\mu_c = C_c (C_1^{-1}\mu_1 + C_2^{-1}\mu_2)$. Substitution to these formulas yields

\begin{eqnarray}
\begin{split}
 \frac{1}{\sqrt{\det(A^TA)}} \frac{1}{z^{D_v}} \mathcal{N}(v;\frac{1}{z}A^{+}x,\frac{\sigma_x}{z^2} (A^TA)^{-1})\mathcal{N}(v;0,C_v) = \\
\frac{1}{\sqrt{\det(A^TA)}} \frac{1}{z^{D_v}} \mathcal{N}(\frac{1}{z}A^{+}x;0,\frac{\sigma_x}{z^2} (A^TA)^{-1} + C_v) \mathcal{N}(v; \mu_c,C_c)
 \end{split} \\
 C_c = (\frac{z^2}{\sigma_x} (A^TA) + C_v^{-1})^{-1} \\
 \mu_c = C_c \frac{z}{\sigma_x} (A^TA) A^{+}x = \frac{z}{\sigma_x} C_c A^{T}x
\end{eqnarray}


\subsection{Rules of differentiation}

Assuming that $y$ and $a$ are vectors and $M$ is a symmetric matrix of appropriate dimension, and $f$ is a scalar function, and $s$ is a scalar variable.

\begin{eqnarray}
\frac{\partial}{\partial y} y^T M y = 2 M y \label{eq:deriv_quadratic} \\
\frac{\partial}{\partial y} a^T y = a \label{eq:deriv_scalarprod} \\
\frac{\partial}{\partial M} y^T M^{-1} y = - M^{-1} yy^T M^{-1} \label{eq:deriv_quad_mat} \\
\frac{\partial}{\partial M} \log \det M = M^{-1} \label{eq:deriv_logdet} \\
\frac{\partial}{\partial s} f(M(s)) = \textrm{Tr} \left[ \frac{\partial f}{\partial M} \frac{\partial M}{\partial s} \right] \label{eq:deriv_chain} \\
\frac{\partial}{\partial s} s M = M \label{eq:deriv_scalar} \\
\frac{\partial}{\partial s} f(s) = f(s) \frac{\partial}{\partial s} \log f(s) \label{eq:deriv_function} \\
\frac{\partial}{\partial M} \log \mathcal{N}(y;a,M) = M^{-1} - M^{-1}(y-a)(y-a)^TM^{-1} \label{eq:deriv_gausscov}
\end{eqnarray}


\subsection{Equivalence of the likelihood formulas}

Expanding \ref{eq:like_gz_alg} yields

\begin{eqnarray}
\begin{split}
p(x \mid z,g) = \frac{1}{\sqrt{\det(A^TA)}} \frac{1}{z^{D_v}} \frac{1}{\sqrt{ (2\pi)^{D_v} \det C(z,g) }} e^{-\frac{1}{2} \left( \frac{1}{z}A^{+}x \right)^T C^{-1}(z,g) \frac{1}{z}A^{+}x}  = \\
= \frac{1}{\sqrt{ (2\pi z^2)^{D_v} \det(A^TA C(z,g))}} e^{-\frac{1}{2z^2} x^TA^{+T} C^{-1}(z,g) A^{+}x}
\end{split} \\
C(z,g) \equiv \frac{\sigma_x}{z^2} (A^TA)^{-1} + C_v
\end{eqnarray}
%
so in the exponent, in the place of the covariance matrix, we have

\begin{equation}
\begin{split}
\frac{1}{z^2} \left( (A^TA)^{-1}A^T \right)^T \left[  \frac{\sigma_x}{z^2} (A^TA)^{-1} + C_v  \right]^{-1} (A^TA)^{-1}A^T = \\
= \frac{1}{z^2} A (A^TA)^{-1} \left[  \frac{\sigma_x}{z^2} (A^TA)^{-1} + C_v  \right]^{-1} (A^TA)^{-1}A^T = \\
= \frac{1}{z^2} A \left[ \left[ \frac{\sigma_x}{z^2} (A^TA)^{-1} + C_v \right] (A^TA) \right]^{-1} (A^TA)^{-1}A^T = \\
= \frac{1}{z^2} A \left[ \frac{\sigma_x}{z^2} I + C_v A^TA \right]^{-1} (A^TA)^{-1}A^T = \\
= \frac{1}{z^2} A \left[ (A^TA) \left[ \frac{\sigma_x}{z^2} I + C_v A^TA \right] \right]^{-1} A^T = \\
= \frac{1}{z^2} A  \left[ \frac{\sigma_x}{z^2} A^TA + A^TA C_v A^TA \right]^{-1} A^T
\end{split}
\end{equation}
%
assuming that $A$ is invertible this is equal to

\begin{equation}
\begin{split}
\frac{1}{z^2} \left[ A^{-1}\right]^{-1}  \left[ \frac{\sigma_x}{z^2} A^TA + A^TA C_v A^TA \right]^{-1} \left[ A^{-T}\right]^{-1} = \\
\frac{1}{z^2} \left[ A^{-1}\right]^{-1}  \left[ A^{-T} \left[ \frac{\sigma_x}{z^2} A^TA + A^TA C_v A^TA \right] \right]^{-1} = \\
\frac{1}{z^2} \left[ A^{-1}\right]^{-1}  \left[ \frac{\sigma_x}{z^2} A + A C_v A^TA \right]^{-1} = \\
\frac{1}{z^2} \left[  \left[ \frac{\sigma_x}{z^2} A + A C_v A^TA\right] A^{-1} \right]^{-1} = \\
\frac{1}{z^2} \left[ \frac{\sigma_x}{z^2} I + A C_v A^T\right]^{-1} = \\
= \left[ \sigma_x I + z^2 A C_v A^T \right]^{-1}
\end{split}
\end{equation}
%
under the square root we have

\begin{equation}
\begin{split}
z^{2 D_v} \det(A^TA C(z,g)) = z^{2 D_v} \det(A^T) \det(A) \det(C(z,g)) = z^{2 D_v} \det(A) \det(C(z,g)) \det(A^T)\\
= z^{2 D_v} \det(A) \det(\frac{\sigma_x}{z^2} (A^TA)^{-1} + C_v) \det(A^T) = 
z^{2 D_v} \det(A) \det(\frac{\sigma_x}{z^2} (A^TA)^{-1} + C_v) \det(A^T) = \\
= z^{2 D_v} \det(A) \det(\frac{\sigma_x}{z^2} A^{-1}A^{-T} + C_v) \det(A^T) = 
z^{2 D_v} \det \left[ A \left[ \frac{\sigma_x}{z^2} A^{-1}A^{-T} + C_v \right] A^T \right] = \\
= z^{2 D_v} \det \left[  \frac{\sigma_x}{z^2} I + A C_v A^T \right] =
\det \left[  \sigma_x I + z^2 A C_v A^T \right]
\end{split}
\end{equation}

\end{document}
