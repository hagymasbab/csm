\documentclass{paper}
\usepackage{amsmath}
\usepackage{amssymb}
\DeclareMathOperator*{\argmin}{arg\,min}
\DeclareMathOperator*{\argmax}{arg\,max}

\begin{document}

\title{Derivative of the posterior of the gestalt model}
\maketitle


\section{Rules of differentiation}

Assuming that $y$ and $a$ are vectors and $M$ is a symmetric matrix of appropriate dimension, and $f$ is a scalar function, and $s$ is a scalar variable.

\begin{eqnarray}
\frac{\partial}{\partial y} y^T M y = 2 M y \label{eq:deriv_quadratic} \\
\frac{\partial}{\partial y} a^T y = a \label{eq:deriv_scalarprod} \\
\frac{\partial}{\partial M} y^T M^{-1} y = - M^{-1} yy^T M^{-1} \label{eq:deriv_quad_mat} \\
\frac{\partial}{\partial M} \log \det M = M^{-1} \label{eq:deriv_logdet} \\
\frac{\partial}{\partial s} f(M(s)) = \textrm{Tr} \left[ \frac{\partial f}{\partial M} \frac{\partial M}{\partial s} \right] \label{eq:deriv_chain} \\
\frac{\partial}{\partial s} s M = M \label{eq:deriv_scalar}
\end{eqnarray}


\section{Form of the full posterior}

\begin{equation}
p(v,g,z \mid x) = p(x \mid v,g,z) p(v \mid g) p(g) p(z) \frac{1}{p(x)} \sim  p(x \mid v,z) p(v \mid g) p(g) p(z)
\end{equation}
%
so the log-posterior will be the following, up to an additive constant, using Gamma priors over $g$ and $z$ defined by shape and scale parameters:

\begin{equation}
\begin{split}
\log p(v,g,z \mid x) \sim \log p(x \mid v,z) + \log p(v \mid g) + \log p(g) + \log p(z) = \\
= \log \mathcal{N}(x;zAv,\sigma_x I) + \log \mathcal{N}(v;0,C_v) + \log \textrm{Gam}(g;sh_g,sc_g) + \log \textrm{Gam}(z;sh_z,sc_z)
\end{split}
\end{equation}
%
logarithms of the used pdfs look as follows:

\begin{eqnarray}
\log  \mathcal{N}(y;\mu,C) = -\frac{1}{2} \left[ \log (2\pi) + \log \det (C) + (y - \mu)^T C^{-1} (y-\mu) \right] \\
\log  \textrm{Gam}(y;sh,sc) = \log(1) - \log(\Gamma(sh)) - sh \log(sc) + (sh-1) \log(y) - \frac{y}{sc}
\end{eqnarray}
%
so discarding all terms that are constant w.r.t. all three variables, the log-posterior is composed as follows:

\begin{equation}
\begin{split}
\log p(v,g,z \mid x) \sim -\frac{1}{2 \sigma_x}(x - zAv)^T (x - zAv) - \\
- \frac{1}{2} \left[ \log \left( \det \left( C_v \right) \right) + v^T C_v^{-1} v \right] + \\
+ \sum_{j=1}^K \left[ (sh_g - 1) \log(g_j) - \frac{g_j}{sc_g} \right] + (sh_z - 1) \log(z) - \frac{z}{sc_z}
\end{split}
\end{equation}
% 
expanding the quadratic form in the first term

\begin{equation}
\begin{split}
(x - zAv)^T (x - zAv) = x^Tx -zv^TA^Tx + z^2v^TA^TAv -zx^TAv = \\
= x^Tx -2zx^TAv + z^2v^TA^TAv
\end{split}
\end{equation}
% 
as $zx^TAv$ is a scalar, thus equal to its transpose. Discarding the term not dependent on any variables of the posterior we get

\begin{equation}
\begin{split}
\log p(v,g,z \mid x) \sim - \frac{z}{2 \sigma_x} \left( zv^TA^TAv - 2x^TAv \right)- \\
- \frac{1}{2} \left[ \log \left( \det \left( C_v \right) \right) + v^T C_v^{-1} v \right] + \\
+ \sum_{j=1}^K \left[ (sh_g - 1) \log(g_j) - \frac{g_j}{sc_g} \right] + \\
+ (sh_z - 1) \log(z) - \frac{z}{sc_z}
\end{split}
\end{equation}

\section{Derivative in $v$}

\begin{equation}
\log p(v,g,z \mid x) \sim - \frac{1}{2} \left[ \frac{z^2}{ \sigma_x} v^TA^TAv - \frac{2z}{ \sigma_x}  x^TAv +  v^T C_v^{-1} v \right]  + f_1(g,z)
\end{equation}
%
lumping the two quadratic forms together

\begin{equation}
\log p(v,g,z \mid x) \sim \frac{z}{ \sigma_x}  x^TAv -  \frac{1}{2} v^T \left[ \frac{z^2}{\sigma_x}A^TA + C_v^{-1} \right] v + f_1(g,z)
\end{equation}
%
Taking the derivative using Eq. \ref{eq:deriv_quadratic} and \ref{eq:deriv_scalarprod} we get

\begin{equation}
\frac{\partial}{\partial v} \log p(v,g,z \mid x) =  \frac{z}{ \sigma_x}  A^Tx - \left[ \frac{z^2}{\sigma_x}A^TA + C_v^{-1} \right] v 
\end{equation}


\section{Derivative in $g$}

\begin{equation}
\begin{split}
\log p(v,g,z \mid x) \sim - \frac{1}{2} \left[ \log \det \left( C_v \right) + v^T C_v^{-1} v \right] + \\
+ \sum_{j=1}^K \left[ (sh_g - 1) \log(g_j) - \frac{g_j}{sc_g} \right]   + f_2(v,z)
\end{split}
\end{equation}
%
Taking the derivative w.r.t. a single $g_i$ using Eq. \ref{eq:deriv_chain} we get

\begin{equation}
\begin{split}
\frac{\partial}{\partial g_i} \log p(v,g,z \mid x) =  - \frac{1}{2} \textrm{Tr} \left[ \frac{\partial}{\partial C_v} \left[ \log \det \left( C_v \right) + v^T C_v^{-1} v \right] \frac{\partial C_v}{\partial g_i}\right]  + \\
+ \frac{\partial}{\partial g_i} \left[ (sh_g - 1) \log(g_i) - \frac{g_i}{sc_g} \right]
\end{split}
\end{equation}
%
using Eq. \ref{eq:deriv_logdet}, \ref{eq:deriv_quad_mat} and \ref{eq:deriv_scalar} we arrive to

\begin{equation}
\frac{\partial}{\partial g_i} \log p(v,g,z \mid x) =  - \frac{1}{2} \textrm{Tr} \left[ \left[ C_v^{-1} - C_v^{-1} v v^T C_v^{-1} \right] C_i \right]  + \frac{sh_g - 1}{g_i} - \frac{1}{sc_g}
\end{equation}


\section{Derivative in $z$}

\begin{equation}
\log p(v,g,z \mid x) \sim - \frac{z}{2 \sigma_x} \left( zv^TA^TAv - 2x^TAv \right) + (sh_z - 1) \log(z) - \frac{z}{sc_z} + f_3(g,v)
\end{equation}

\begin{equation}
\frac{\partial}{\partial z} \log p(v,g,z \mid x) =  \frac{1}{\sigma_x} \left[ x^TAv - z v^T A^TA v \right] + \frac{sh_z - 1}{z} - \frac{1}{sc_z}
\end{equation}

\end{document}
