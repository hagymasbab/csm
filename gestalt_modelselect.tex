\documentclass{paper}
\usepackage{amsmath}
\usepackage{amssymb}
\DeclareMathOperator*{\argmin}{arg\,min}

\begin{document}

\title{Model selection for the gestalt model}
\maketitle

\section{The likelihood function}

The full likelihood looks like this for a set of $N$ images

\begin{equation}
\begin{split}
p(X \mid V,G,Z,C_{1..k}) = \prod_{n=1}^N p(x_n \mid v_n,z_n)
\end{split}
\end{equation}
%
marginalising over all the latents will produce

\begin{equation}
p(X \mid C_{1..k}) = \iiint_{-\infty}^{\infty} \prod_{n=1}^N p(x_n \mid v_n,z_n) p(v_n \mid g_n) p(g_n) p(z_n) \mathrm{d}V\mathrm{d}G\mathrm{d}Z
\end{equation}
%
taking $NL$ samples from the priors of all latents, we can approximate this as

\begin{equation}
p(X \mid C_{1..k}) \approx \frac{1}{L} \sum_{l = 1}^{L} \prod_{n=1}^N p(x_n \mid v_{n,l},z_{n,l}) 
\end{equation}

%\begin{equation}
%\begin{split}
%p(X \mid C_{1..k}) = \prod_{n=1}^N \iiint_{-\infty}^{\infty} p(x_n \mid v_n,z_n) p(v_n \mid g_n) p(g_n) p(z_n) \mathrm{d}v_n\mathrm{d}g_n\mathrm{d}z_n = \\
%= \prod_{n=1}^N \iint_{-\infty}^{\infty}  p(g_n) p(z_n) \int_{-\infty}^{\infty} p(x_n \mid v_n,z_n) p(v_n \mid g_n) \mathrm{d}v_n\mathrm{d}g_n \mathrm{d}z_n
%\end{split}
%\end{equation}
%%
%the product of the two conditionals can be rewritten as a Gaussian over $v$ times another Gaussian
%
%\begin{equation} 
%p(x \mid v,z) p(v \mid g) = \frac{1}{\sqrt{\det(A^TA)}} \frac{1}{z^{D_v}} \mathcal{N}(\frac{1}{z}A^{+}x;0,\frac{\sigma_x}{z^2} (A^TA)^{-1} + C_v) \mathcal{N}(v; \mu_c,C_c)
%\end{equation}
%%
%where $v$ only appears in the second Gaussian, so integrating this formula will set that term to 1 and leave everything else as it is.
%
%\begin{equation}
%p(X \mid C_{1..k}) =  \det(A^T A)^{-\frac{N}{2}} \prod_{n=1}^N \int_{0}^{\infty} p(z) \frac{1}{z^{D_v}}  \int_{0}^{\infty}  p(g) \mathcal{N}(\frac{1}{z} A^{+}x; 0,\frac{\sigma_x}{z^2} (A^T A)^{-1} + C_v) \mathrm{d}g \mathrm{d}z
%\end{equation}
%
%\section{Approximation with sampling}
%
%The integrals may be approximated by averaging over samples from $p(g)$ and $p(z)$, leaving out terms constant in $g$, $z$ and $x$
%
%\begin{equation}
%p(X \mid C_{1..k}) \sim \prod_{n=1}^N  \frac{1}{L^2} \sum_{l_1=1}^L \frac{1}{z_{l_1}^{D_v}}  \sum_{l_2=1}^L \mathcal{N}(\frac{1}{z_{l_1}} A^{+}x_n; 0,\frac{\sigma_x}{z_{l_1}^2} (A^T A)^{-1} + C_v^{l_2})
%\end{equation}
%% 
%as $z$ and $g$ are independent, we can choose a large enough $L$ and merge the sums, and bring the division by $L$ outside of the product
%
%\begin{equation}
%p(X \mid C_{1..k}) \sim \frac{1}{L^N} \prod_{n=1}^N  \sum_{l=1}^L \frac{1}{z_{l_1}^{D_v}}  \mathcal{N}(\frac{1}{z_l} A^{+}x_n; 0,\frac{\sigma_x}{z_l^2} (A^T A)^{-1} + C_v^l)
%\end{equation}
%%
%taking the logarithm we get
%
%\begin{equation}
%\log p(X \mid C_{1..k}) \sim - N \log(L) + \sum_{n=1}^N \log \left[ \sum_{l=1}^L \frac{1}{z_{l_1}^{D_v}}  \mathcal{N}(\frac{1}{z_l} A^{+}x_n; 0,\frac{\sigma_x}{z_l^2} (A^T A)^{-1} + C_v^l) \right]
%\end{equation}
%
\end{document}
